\documentclass[../../document.tex]{subfiles}
\begin{document}

\section*{Aufgabe 3}

\subsection*{a)}

\begin{equation*}
    \begin{split}
        \ubar{Z\cind{m}} &= R + j \omega * L \Rightarrow \sqrt[]{R^2 + \omega^2 * L^2} * \upe^{j \arctan{\frac{\omega * L}{R}}} = \sqrt[]{R^2 + \omega^2 * L^2} * \upe^{j \varphi}\\
        \hat{i}\cind{N} &= \frac{\hat{u}\cind{N}}{\ubar{Z\cind{m}}} = \frac{\sqrt[]{2} * \SI{115}{\volt} * \upe^{j \varphi}}{\sqrt[]{R^2 + \omega^2 * L^2} * \upe^{j \varphi}} = \frac{\sqrt[]{2} * \SI{115}{\volt}}{\sqrt[]{R^2 + \omega^2 * L^2}} * \upe^{-j \varphi}\\
    \end{split}
\end{equation*}

\subsection*{b)}

\subsubsection*{Scheinleistung}

\begin{equation*}
    \begin{split}
        \ubar{S} &= \frac{1}{2} * \hat{\ubar{u}} * \hat{\ubar{i}}^*\\
        &= \frac{1}{2} * \sqrt[]{2} * \SI{115}{\volt} * \frac{\sqrt[]{2} * \SI{115}{\volt}}{\sqrt[]{R^2 + \omega^2 * L^2}} * \upe^{+j \varphi} = \frac{1}{2} * \SI{115}{\volt} * \frac{\SI{115}{\volt}}{\sqrt[]{R^2 + \omega^2 * L^2}} * \upe^{j \varphi}\\
        &\Rightarrow \frac{{U\cind{eff}}^2}{ \left\lvert Z \right\rvert } * \upe^{j \varphi}\\
        S &= \left\lvert S \right\rvert = \frac{{U\cind{eff}}^2}{ \left\lvert Z \right\rvert } = \frac{{\SI{115}{\volt}}^2}{ \sqrt[]{R^2 + \omega^2 * L^2 }} \left[VA\right]\\
    \end{split}
\end{equation*}

\subsubsection*{Wirkleistung}

\begin{equation*}
    \begin{split}
        P &= \frac{{U\cind{eff}}^2}{Z} * \cos{\varphi} = \frac{{\SI{115}{\volt}}^2}{ \sqrt[]{R^2 + \omega^2 * L^2 }} * \cos{\arctan{\frac{\omega * L}{R}}} \left[W\right]\\
    \end{split}
\end{equation*}

\subsubsection*{Blindleistung}

\begin{equation*}
    \begin{split}
        Q &= \frac{{U\cind{eff}}^2}{Z} * \sin{\varphi} = \frac{{\SI{115}{\volt}}^2}{ \sqrt[]{R^2 + \omega^2 * L^2 }} * \sin{\arctan{\frac{\omega * L}{R}}} \left[var\right]\\
    \end{split}
\end{equation*}

\subsubsection*{Leistungsfaktor}

\begin{equation*}
    \begin{split}
        \lambda &= \frac{P}{S} = \cos{\varphi} = \cos{\arctan{\frac{\omega * L}{R}}}\\
    \end{split}
\end{equation*}

\subsection*{c)}

In manchen Teilen der Welt (z.B. die USA/Japan) wird die Frequenz $f = \SI{60}{\hertz}$ verwendet. Dort könnte der Motor eingesetzt werden.

\end{document}